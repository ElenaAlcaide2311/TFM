%%
%% mxlyrdoc.tex:  Manual for the MusiXTeX lyrics package musixlyr
%%
%% Copyright (C) 1996-2003  Rainer Dunker
%%
%% This document is free software; you can redistribute it and/or modify
%% it under the terms of the GNU General Public License as published by
%% the Free Software Foundation; either version 2 of the License, or
%% any later version.
%%
%% This document is distributed in the hope that it will be useful,
%% but WITHOUT ANY WARRANTY; without even the implied warranty of
%% MERCHANTABILITY or FITNESS FOR A PARTICULAR PURPOSE.  See the
%% GNU General Public License for more details.
%%
%% You should have received a copy of the GNU General Public License
%% along with this document; if not, write to the Free Software
%% Foundation, Inc., 675 Mass Ave, Cambridge, MA 02139, USA.
%%
%% Author:
%%  Rainer Dunker
%%  Wachtelweg 31
%%  85591 Vaterstetten
%%  Germany
%%
%%  E-mail:  rainer.dunker@web.de
%%
\documentclass[twoside]{article}
\usepackage{musixtex,multicol,makeidx}

\input musixlyr

\textwidth210mm \advance\textwidth-2in               % These format settings
\textheight297mm \advance\textheight-2in             % can be freely changed!
\advance\textheight-\headheight \advance\textheight-\headsep
\footskip0pt \topmargin0pt \evensidemargin0pt \oddsidemargin0pt

\pagestyle{headings}
\parskip0.5ex plus 0.25ex minus 0.25ex
\parindent0pt
\sloppy

\nobarnumbers
\smallmusicsize

%
% command index
%
\makeatletter
\renewenvironment{theindex}
  {\let\item\@idxitem
   \begin{multicols}{2}
   \parskip0pt}
  {\end{multicols}}
\def\printcmd#1{{\tt\char92 #1}}
\def\ci#1{\printcmd{#1}\bci{#1}}
\def\bci#1{\index{#1@\printcmd{#1}}} % "blind" command index entry
\makeindex

\let\myqnwidth\qn@width
\makeatother

\def\musixlyr{{\tt musixlyr}}
\def\PMX{{\bf PMX}}
\def\var#1{\hbox{$\langle$#1$\rangle$}}
\def\oneversespace{\par\vspace*{2mm}}
\def\twoversesspace{\par\vspace*{7mm}}




\begin{document}


\title{\musixlyr\ --- a \musixtex\ Extension Package for Lyrics Handling
       (Version 2.1c)}
\author{\copyright\ Rainer Dunker 1996--2003\\
	{\tt rainer.dunker@web.de}}
\maketitle
\thispagestyle{empty}


{\small\sl (The page layout of this document is tailored to A4 paper
 format, but the text width and height can be freely changed in order
 to meet other paper sizes; no part of the layout relies on a
 particlular size.)\/}

{\parskip0pt\tableofcontents}



\section{Introduction}


\subsection{Basic concepts}

\musixlyr\ is a set of \TeX\ macros to be used with Taupin \musixtex\
(version T.52 or later) for typesetting vocal music. Its purpose is
to compensate two drawbacks of \musixtex's lyrics handling:
\begin{itemize}
 \item Typesetting lyrics with the ``native'' \musixtex\ commands
   \verb|\zcharnote|, \verb|\zsong| etc.\ tends to be quite
   inefficient, particularly if the lyrics have to be changed or
   corrected. The idea underlying \musixlyr\ is to separate lyrics
   coding from music coding and let \TeX\ weave them together with as
   little manual interference as possible. As a result you can enter
   and edit lyrics (nearly) as easily as normal text.
 \item \musixtex\ has no built-in mechanism for centering hyphens between
   syllables and for handling hyphenation at long melismas. This is
   implemented in \musixlyr\ following the example of engraved music.
\end{itemize}



\subsection{New since version~2.0}

\musixlyr~2.0 provides the following major enhancements:
\begin{itemize}
 \item Support for lyrics on multi-staff instruments
   $\longrightarrow$~section~\ref{multi-staff-instruments}
 %
 \item Melisma declaration as part of the lyrics definition,
   i.\,e.~within \ci{setlyrics}
   $\longrightarrow$~section~\ref{melismas}
 %
 \item Improved scheme for vertical lyrics positioning
   $\longrightarrow$~section~\ref{vertical-positioning}
 % 
 \item Facility for posting lyrics anywhere on a staff, even where no
   notes are present at all
   $\longrightarrow$~section~\ref{lyric*-verse-numbers}
\end{itemize}

These new features are especially interesting when \musixlyr\ is used in
conjunction with \PMX:
\begin{itemize}
 \item Automatic detection of upper/lower voice context in
   a two-voice staff; explicitly addressing lyrics commands to
   auxiliary lyrics with \ci{auxlyr} is no longer required
   $\longrightarrow$~section~\ref{auxlyr-attachment}
 %
 \item Context-dependent lyrics settings that do not need to be
   adressed to absolute instrument/staff numbers, making lyrics
   assignment robust against part extraction
   $\longrightarrow$~section~\ref{assignment-by-context}
\end{itemize}

Finally, a few improvements have been made that do not bear explicit
new features but work rather behind the scenes:
\begin{itemize}
 \item No more \TeX\ register allocations for each verse name newly
   defined by \ci{setlyrics} or \ci{copylyrics}; the use
   of those registers has been replaced by macros.
 %
 \item Due to the abandoned per-verse register allocations, the
   \ci{enableauxlyrics} command is obsolete. It is retained for the
   sake of backward compatibility, but it has no effect.
 %
 \item Slightly more robust processing of undefined verses. An error
   is raised immediately when an undefined verse is assigned to a
   staff/instrument, and the undefined verse name is implicitly
   defined as having empty lyrics. This avoids obscure errors in later
   stages of \TeX\ processing.
\end{itemize}

The current version~2.1c is mainly a bugfix version of~2.0, with a few
functional additions:
\begin{itemize}
 \item Possibility to add additional lyrics to an existing lyrics
   definition without discarding the remainder of the current lyrics
   line contents; useful if lyrics are defined in several portions
   spread over the score (see~\ref{appendlyrics}).
 \item The \musixlyr\ version is identified by the predefined macro
   \verb|\musixlyrversion|, enabling other \TeX\ code to check for
   version compatibility (according to a suggestion by Dirk Laurie).
\end{itemize}

Suggestions for other useful things are welcome.



\subsection{Backward incompatibility issues}

The following properties make \musixlyr\ versions~2.$x$ incompatible
to earlier versions:
\begin{itemize}
 \item
   It is no longer possible to represent ``empty syllables'' by
   empty \TeX\ environments `\verb|{}|' within \ci{setlyrics}; see
   also section~\ref{empty-syllable}.
 %
 \item
   Within \ci{setlyrics}, the underscore character has got a
   special meaning and thus can no longer be used as usual; see
   section~\ref{melismas-by-underscores}.
 %
 \item
   By default, auxiliary lyrics are now placed vertically {\em
   above\/} (no longer below) the staff they are assigned to; see
   section~\ref{auxlyr-vertical-position-default}.
\end{itemize}


The following bugfix of \musixlyr~2.1b also implies a slight backward
incompatibility:
\begin{itemize}
 \item
   At the beginning of a music line (system), lyrics have often been
   right-shifted because \musixlyr\ has assumed MusiXTeX's horizontal
   ``zero'' position as the left boundary for lyrics positioning. Now,
   lyrics are allowed to extend into the left margin as far as
   required for centered placement under/above the note.

   Although this is a more ``natural'' behaviour, it may lead to
   collisions between lyrics and other stuff (like dynamic marks) that
   have not occurred using older \musixlyr\ versions. In such a case,
   there are two ways to solve the problem:
   %
   \begin{itemize}
     \item Restore the former \musixlyr\ behaviour globally by saying
       \begin{quote}
	 \ci{oldlyrlinestart}
       \end{quote}
       near the top of your score source.
       %
     \item Right-shift the colliding lyrics where necessary using
       \ci{lyroffset} (see~\ref{lyroffset}).
   \end{itemize}
 \end{itemize}



\subsection{Restrictions}

\begin{itemize}
 \item \musixlyr\ does not influence the \musixtex/{\tt musixflx}
   spacing algorithm in order to make the notes spacing dependent on
   the space requirements of the lyrics. See section~\ref{horizpos}
   about how spacing problems are handled.
 %
 \item \musixlyr\ seems to work correctly in my preferred working
   environment, but I cannot exclude problems in situations I have
   not undergone (especially concerning the use of \musixtex\
   preprocessors, the \musixtex\ extension library, and past or
   future \musixtex\ releases). Thus it is important that you point out
   encountered bugs and incompatibilities to me; there is little
   chance that I notice them on my own.

   Please do also bother me if you find this manual --- or parts of
   it --- incomprehensible or incomplete.
\end{itemize}



\section{Installation}

Put the {\tt musixlyr.tex} file into a directory that is included in
your \verb|TEXINPUTS| search path --- that's all. (Refer to your local \TeX\
documentation for details on this.)



\section{Usage}

Before using \musixlyr\ macros, state
\begin{quote}\verb|\input musixlyr|\end{quote}
near the beginning of your source file but {\bf after} the
\verb|\input musixtex| (\LaTeX: \verb|\usepackage{musixtex}|)
command.



\subsection{Entering lyrics}


\subsubsection{Lyrics as lines of words}

As opposed to the ``vertical'' music construction of
\musixtex, \musixlyr\ treats lyrics ``horizontally'' as {\em lyrics
lines.\/} Each separate ``thread'' of words is a separate lyrics
line; thus, in the following example\footnote{Excerpt from
``Tourdion'' by {\sc Pierre Attaignant} ({\sl ars
musica\/} vol.\,4, M\"oseler Verlag Wolfenb\"uttel/Z\"urich)},

\begin{music}
 \setlyrics{soprano}
	   {Quand je bois du vin clai-ret, a-mi tout}
 \setlyrics{alto}{Le bon vin nous}
 \setlyrics{tenor}{Bu-vons bien, bu-vons mes a-mis,}
 \copylyrics{tenor}{bass}
 \assignlyrics4{soprano}
 \assignlyrics3{alto}
 \assignlyrics2{tenor}
 \assignlyrics1{bass}
 \setsongraise3{-1.5mm}
 \small
 \instrumentnumber4
 \generalsignature2
 \generalmeter{\meterfrac32}
 \setclef1\bass
 \songtop4
 \songbottom1
 \sepbarrules
 \startextract
  \NOtes\wh{`e}\sk\sk\sk\ha e&%
	\wh g\sk\sk\sk\ha e&%
	\wh b\sk\sk\sk\ha b&%
	\qa{efg'a!gf}\en\bar
  \NOTesp\wh{`e}&\wh{'b}&\wh e&\hup e\en
  \NOtes\sk\ha{`e}&\sk\ha{'b}&\sk\ha e&\qa{fg'a}\en
  % force hyphens of incomplete words:
  \def\atnextbar{\znotes\lyric{}&\lyric{}\en}%
 \endextract
\end{music}
\oneversespace

there are four lyrics lines notwithstanding that two of them consist
of the same words.

Note that the number of lyrics lines does not depend on the number
of staves: Any number of lyrics lines (including none) can be attached
to each staff (more about multiple lyrics lines per staff in
section~\ref{multilyrlines}).


\subsubsection{Setting up a lyrics line}

A lyrics line is set up using the command
\begin{quote}
  \ci{setlyrics}\verb|{|\var{name}\verb|}{|\var{words}\/\verb|}|
\end{quote}
where \var{name} is a name\footnote{This name must not contain
backslashes or commas. Most other characters do not cause problems;
however, it is safer to use only letters and digits.} for the lyrics
line (e.\,g.\ the name of the voice singing these lyrics) and
\var{words} is the complete text where multi-syllable words must be
given {\bf hyphenated.} For example, the soprano lyrics of the above
example are set as follows:
\begin{quote}
 \bci{setlyrics}
 \begin{verbatim}
\setlyrics{soprano}{Quand je bois du vin clai-ret, a-mi tout}\end{verbatim}
\end{quote}

Important notes on using \ci{setlyrics}:
\begin{itemize}
  \item
    Within the lyrics definition, spaces and hyphens are used to
    split the whole text into words or syllables. If you want a space
    or hyphen appear {\em within\/} a word/syllable, use the command
    \verb|\space| or \ci{lyrhyphenchar}, respectively, instead.
  \item
    \label{empty-syllable}%
    If you want the lyrics attachment to skip over a note,
    i.\,e.~a certain note shall not receive a syllable, you may
    achieve this by inserting an empty syllable in the \var{words}
    text. However, neither a ``really empty'' syllable (i.\,e.~two
    consecutive separating spaces) nor an empty environment \verb|{}|
    will work, but you may use either \verb|{\empty}| or
    \verb|\empty{}| for this.

    Normally, the \ci{nolyr} command (discussed later in this
    manual) should be used for this purpose, but an empty syllable may
    be more handy if it occurs in only one of multiple assigned
    verses; see also sections~\ref{melismas-by-underscores}
    and~\ref{multilyrlines}.
  \item
    Note that, since lyrics lines are cut into pieces which are
    processed separately, you cannot apply things like temporary font
    changes in the usual manner (i.\,e.~using \TeX\
    environments). See section~\ref{layoutpar} about how to accomplish
    layout changes.
  \item
    Applying \ci{setlyrics} to an already defined \var{name}
    redefines the lyrics line. This may be useful at very long pieces
    because setting up and applying lyrics in smaller portions
    (e.\,g.\ movement by movement) saves both time and memory.
  \item
    It is possible to specify melismas --- that shall occur later when
    a verse is attached to the music --- already within
    \ci{setlyrics}; see section~\ref{melismas-by-underscores} for
    details.

\end{itemize}

There are two variants of the \verb|\setlyrics| command available:
\begin{itemize}
  \item
    If several voices have the same lyrics, you do not have to copy
    and paste the lyrics for each further voice; just say
    \begin{quote}\ci{copylyrics}\verb|{|\var{from name}%
			       \verb|}{|\var{to name}\verb|}|\end{quote}
    to set up a lyrics line named \var{to name} with the same contents
    as \var{from name}.
  \item
    Applying \ci{setlyrics} to an already defined \var{name}
    has the effect of discarding all previous contents of \var{name},
    even if part of the lyrics is still pending for being
    processed. If you want to {\em append\/} further lyrics to an already
    existing lyrics line instead --- without interrupting the
    processing of the current lyrics contents ---, use the command
    \label{appendlyrics}
    \begin{quote}
      \ci{appendlyrics}\verb|{|\var{name}\verb|}{|\var{words}\/\verb|}|
    \end{quote}
    instead of \verb|\setlyrics|.
    % A warning about performance loss due to \appendlyrics's material
    % accumulation may be added here; it should occur only if \golyr
    % jumps to or near to the lyrics beginning are performed.
\end{itemize}


\subsubsection{Assigning lyrics lines to staves}

{\bf Note:} During the following discussion, the term {\em staff\/} is
used wherever \musixtex\ would precisely use the term {\em
instrument\/} instead, assuming that lyrics are posted to single-staff
instruments only. However, everything also applies to the assignment
of lyrics to staves of multi-staff instruments; see
section~\ref{multi-staff-instruments} about how to use lyrics in a
multi-staff instrument context.

To assign a lyrics line to a staff, say
\begin{quote}\ci{assignlyrics}\verb|{|\var i%
	     \verb|}{|\var{name}\verb|}|\end{quote}
where \var i is the instrument number (counted as usual from bottom to
top) and \var{name} the name of the lyrics line.

The resulting assignment can be redefined everywhere. This is useful
if the arrangement of voices on staves changes within the score; when
a voice is continued on a different staff at the next system, use
\ci{assignlyrics} (preferrably enclosed in \verb|\atnextline{...}|) to
make the respective lyrics line follow this change without
interrupting the flow of words.

If a lyrics line is assigned to instrument \var i and you want it to be
continued on a different instrument while assigning no other lyrics line
to instrument \var i, you must explicitly cancel the former assignment by
saying \ci{assignlyrics}\verb|{|\var i\verb|}{}|.

{\bf Caution:} Never assign a lyrics line to more than one
staff simultaneously; if the same lyrics are needed at
several staves in parallel, use \ci{copylyrics} (see above).

As a summary of this section, here is the complete ``lyrics
definition section'' of the above example:
\begin{quote}
 \bci{setlyrics}
 \bci{copylyrics}
 \bci{assignlyrics}
 \begin{verbatim}
\setlyrics {soprano}{Quand je bois du vin clai-ret, a-mi tout}
\setlyrics {alto}   {Le bon vin nous}
\setlyrics {tenor}  {Bu-vons bien, bu}
\copylyrics{tenor}  {bass}
\assignlyrics4{soprano}
\assignlyrics3{alto}
\assignlyrics2{tenor}
\assignlyrics1{bass}\end{verbatim}
\end{quote}

See also section~\ref{assignment-by-context} about an alternative way
to assign lyrics to staves.



\subsection{Managing the attachment of lyrics to notes}
\label{manipulation}


\subsubsection{Automatic lyrics attachment}
\label{auto-attachment}

By default, the syllables of a lyrics line are consecutively attached to
all {\em spacing\/} notes of the assigned staff. Chords thus receive only
one syllable since they usually contain exactly one spacing note.
However, this may be different if you are using auxiliary lyrics;
see section~\ref{auxlyrics} for details.

``Spacing'' in this context means: generated by one of the commands
\verb|\qu|, \verb|\hu|, \verb|\wh|, etc.\ and not by \verb|\zqu|,
\verb|\zhu|, \verb|\zq|, etc. Spacing or not does {\em not\/} depend on
whether the note-generating command is enclosed in \verb|\znotes| or
any other \verb|\notes| command. So if any note seems to have been
left out by the attachment automatism, first make sure that this
note is generated using a command not beginning with \verb|\z|.

If the number of words of a lyrics line gets exhausted before all
notes of the respective staff are processed, ``???'' is posted at all
remaining notes.


\subsubsection{Addressing manipulations to lyrics}

In the ``ideal'' case, the attachment automatism perfectly distributes the
lyrics over the music. In practice, however, situations in which the
automatism must be ``helped'' manually are frequent. Such
manipulations must be posted within the music code. Here are the
basic rules:
\begin{enumerate}
 \item All lyrics manipulations must be stated inside
   \verb|\notes...\enotes|.
   (Exception: melisma specifications within \verb|\setlyrics|, see
   section~\ref{melismas-by-underscores}).
 \item Lyrics manipulations solely refer to the lyrics line(s)
   attached to the current staff (``current'' with respect to the
   enclosing `\verb|&|' characters).
 \item Similar to many other \musixtex\ commands (e.\,g.\ for beams
   and slurs), lyrics manipulations must be issued before the
   corresponding note commands.
\end{enumerate}
Examples in the following sections will illustrate the application.


\subsubsection{Melismas}
\label{melismas}

Regarding lyrics and music separately, melismas are no property of
either one; instead, they characterize the relation between both and
play a role only where music and lyrics occur together. Using
\musixlyr, however, lyrics coding and music coding are basically
unrelated, but melismas do have to be specified somewhere. You have to
decide yourself whether you regard melismas as an attribute rather of
the lyrics or rather of the music (this may vary according to the type
of score you are preparing, or may even differ within a score);
\musixlyr\ provides two corresponding styles of melisma specification
which are discussed in the following paragraphs.

\begin{description}
 \item[Melismas as a music property.]
    Within the music code, a melisma is initiated and terminated by the
    command pair
    \begin{quote}
     \hspace*{\fill}\ci{beginmel}\hfill\ci{endmel}\hspace*{\fill}
    \end{quote}
    issued before the first and last note, respectively, of the melisma.
    This left-justifies the current word/syllable, suspends lyrics until
    after the melisma, and sets up a ``word extension
    underline''\footnote{I am not sure whether this is the appropriate
    term; it is taken from the {\sl Finale\/} reference manual.} unless
    the current word is continued afterwards and thus has to be
    hyphenated. For example,
    
    \begin{music}
     \resetlyrics
     \instrumentnumber1
     \small\scale{0.6}
     \setlyrics{melismatic}{a me-lis-ma}
     \assignlyrics1{melismatic}
     \startextract
      \NOtes\beginmel\qa{def}\endmel\qa e\en\bar
      \NOtes\beginmel\qa{def}\endmel\qa g\en\bar
      \NOtes\beginmel\qa{'a!gf}\endmel\qa e\en\bar
      \NOtes\beginmel\qa{defe}\en\bar
      \NOTEs\endmel\wh d\en
     \endextract
    \end{music}
    \oneversespace
    
    has been coded as:
    \begin{quote}
     \bci{beginmel}
     \bci{endmel}
     \begin{verbatim}
...
\NOtes\beginmel\qa{def}\endmel\qa e\en\bar
\NOtes\beginmel\qa{def}\endmel\qa g\en\bar
\NOtes\beginmel\qa{'a!gf}\endmel\qa e\en\bar
\NOtes\beginmel\qa{defe}\en\bar
\NOTEs\endmel\wh d\en
...\end{verbatim}
    \end{quote}
    If a system break occurs within the melisma, hyphenation or underline
    is continued at the next system, thus automatic line breaking is
    applicable.

 \item[Melismas as a lyrics property.]
   \label{melismas-by-underscores}
   To specify a melisma within a lyrics definition (i.\,e.~within
   \ci{setlyrics} or \ci{appendlyrics}), simply append an underscore
   character to a syllable to make it span over an additional note,
   two underscores to span over two additional notes, and so
   on. Alternatively, you may append to a syllable a single underscore
   followed by a number which denotes the number of {\em additional\/}
   melisma notes: \verb|abc_| is equivalent to \verb|abc_1|,
   \verb|abc__| to \verb|abc_2|, and so on.
   
   No matter which of the two underscore appendix styles you use: If a
   melismatic syllable is hyphenated to the following one, the melisma
   denotation must {\bf precede} the hyphen --- for example:
   \verb|abc__-def| ({\bf not:} \verb|abc-__def|).
   
   All melisma properties mentioned above (lyrics suspension, left
   justification of syllable, word extension underline, system break
   support) also apply to melismas specified this way. For reference,
   here is the lyrics definition section of the example above, this time
   utilizing the underscore melisma technique (note that, in this case,
   all \ci{beginmel} and \ci{endmel} commands must be omitted in the
   music section):
   
   \begin{quote}
    \ci{setlyrics}\verb|{melismatic}{a___ me_3-lis___-ma_4}|
   \end{quote}

 \item[Comparing both points of view.]
   The music-oriented melisma view (using \ci{beginmel}/\ci{endmel}) is
   advantageous for music with a rather loose binding to certain lyrics,
   e.\,g.~folk songs that are to be printed with various verse compilations
   or in different languages. You may then enter each verse without
   taking care of melismas because they are coded just once as part of
   the music.
   
   On the other hand, the lyrics-focused melisma style (using
   underscores) is best suited to pieces with a fixed lyrics-to-music
   binding; it is easier to type and to read, but less flexible as
   regards combining the same lyrics with different music. It is
   especially advantageous for multiple verses (see
   section~\ref{multilyrlines}) with different syllables-to-notes
   distribution (see verse~3 of the ``Drunken sailor'' example in
   section~\ref{versenumbers}); it is much easier to code each verse's
   melismas individually than tracking all melismas in parallel using
   the \ci{verses} command (see again section~\ref{multilyrlines}).

 \item[Note.]
   You may use both styles within the same score but you should avoid
   activating both melisma types simultaneously at the same lyrics line;
   the result of such an occurrence is undefined.
\end{description}


\subsubsection{Turning automatic lyrics attachment on/off}

If the attachment automatism is unsuitable for some
situation --- e.\,g.\ at complex polyphonic voicings or wherever you
want to place all syllables manually --- you can turn it off using
\begin{quote}\ci{lyricsoff}\end{quote}
after the last note that shall get a syllable. (If your piece begins
with a lyrics-less section, state \ci{lyricsoff} before the first
note of a staff.) To re-enable automatic lyrics attachment, use
\begin{quote}\ci{lyricson}\end{quote}
before the first note that shall get a syllable.

If automatic lyrics attachment shall skip over certain single notes
only, the \ci{nolyr} command is more suitable (see
section~\ref{nolyr-command}).


\subsubsection{Lyrics repetition and skipping}

As in the following example of a polyphonic piece,

\begin{music}
 \resetlyrics
 \instrumentnumber2
 \setclef1\bass
 \songtop2
 \songbottom1
 \small
 \setlyrics{amen1}
	   {\llabel{amenlabel}A-men!} \assignlyrics1{amen1}
 \copylyrics{amen1}{amen2}            \assignlyrics2{amen2}
 \startextract
  \NOtes% lower staff:
	  \itenu0K\beginmel\wh K\sk\sk\sk
	  \midslur2\tten0\endmel\hlp K&%
	% upper staff:
	  \qa f\lyrnop\lyrpt,\qa g%
	  \golyr{amenlabel}\lclyr\qa h\lyrnop\lyrpt,\qa{_i}%
	  \golyr{amenlabel}\lclyr\beginmel\islurd1h\qa{hg}%
	  \tslur1f\endmel\qa f\en
  \NOTEs\wh N&\wh g\en
 \endextract
\end{music}
\oneversespace

it often happens that some lyrics phrases are repeated by some but
not all voices, or that each voice has its individual arrangement of
lyrics repetitions. Nevertheless, you do not have to enter
these repetitions directly within \ci{setlyrics}; instead, you
enter the lyrics as if they were performed in an unbroken
sequence. Repetitions and skips are then handled using a ``go to''
technique involving the following steps:
\begin{description}
 \item[Labelling ``go to'' targets.] A syllable which has to be jumped at
   must be preceded (without spaces in between) by
   \begin{quote}\ci{llabel}\verb|{|\var{label name}\verb|}|\end{quote}
   within \ci{setlyrics}.
 \item[Performing jumps.] Within the music code, say
   \begin{quote}\ci{golyr}\verb|{|\var{label name}\verb|}|\end{quote}
   before the note at which the jump has to be performed.

   {\bf Caution:} If you try to jump to an undefined lyrics label, no
   meaningful error message is produced. However, if \TeX\ reports an
   error at a \ci{golyr} command, you should check conformity of
   defined and used labels.
\end{description}
The following actions must be taken in special cases:
\begin{description}
 \item[Adding punctuation marks.] Usually a comma or period is
   appended to the word preceding a jump; this is done by the command
   \begin{quote}\ci{lyrpt}\var{punctuation char}\end{quote}
   where \var{punctuation char} can be any single punctuation
   character.
 \item[Removing punctuation marks.] Before adding a punctuation mark
   to a word as said above, an already existing punctuation mark
   should be removed by the command:
   \begin{quote}\ci{lyrnop}\end{quote}
 \item[Lower case letters.] Words at jump targets often begin with a
   capital letter which should appear lower case after a jump;
   this can be forced by the command:
   \begin{quote}\ci{lclyr}\end{quote}
\end{description}
Here is a code excerpt from the above example, summarizing the use of
all mentioned commands:
\begin{quote}\small
 \bci{llabel}
 \bci{copylyrics}
 \bci{setlyrics}
 \bci{lyrnop}
 \bci{lyrpt}
 \bci{golyr}
 \bci{lclyr}
 \bci{beginmel}
 \bci{endmel}
 \begin{verbatim}
\setlyrics{amen1}
          {\llabel{amenlabel}A-men!} \assignlyrics1{amen1}
\copylyrics{amen1}{amen2}            \assignlyrics2{amen2}
\startextract
 \NOtes% lower staff:
         \itenu0K\beginmel\wh K\sk\sk\sk
         \midslur2\tten0\endmel\hlp K&%
       % upper staff:
         \qa f\lyrnop\lyrpt,\qa g%
         \golyr{amenlabel}\lclyr\qa h\lyrnop\lyrpt,\qa{_i}%
         \golyr{amenlabel}\lclyr\beginmel\islurd1h\qa{hg}%
         \tslur1f\endmel\qa f\en
 \NOTEs\wh N&\wh g\en
\endextract\end{verbatim}
\end{quote}
Note that the ``go to'' technique is recommendable mainly if different
voices perform different repetitions or skips on the same lyrics; if
all voices perform the same sequence of repetitions, however, it is
easier to hard-code them within \ci{setlyrics}.


\subsubsection{Miscellaneous}
\label{miscmanip}

\begin{description}
 \item[Excluding single notes.]
   \label{nolyr-command}%
   In the following situation,

   \begin{music}
    \resetlyrics
    \instrumentnumber1
    \setlyrics1{ca-den-za}
    \assignlyrics11
    \startextract
     \NOtes\zqlp{'c}\qu e\en
     \Notes\qu{'d}\nolyr\cl b\en
     \bar
     \NOTes\zhl{'c}\hu c\en
    \endextract
   \end{music}
   \oneversespace

   the eighth note must be excluded from receiving a syllable. This
   can be accomplished saying
   \begin{quote}\ci{nolyr}\end{quote}
   before the command generating this note.
 \item[Manual lyrics posting.] When automatic lyrics attaching is
   switched off, a word or syllable can be placed manually saying
   \begin{quote}\ci{lyr}\end{quote}
   before the command generating the note that shall get this
   word/syllable.

   This command also works independently from a following note command;
   you can issue it before a rest, a \verb|\sk| or even alone for
   itself in order to place a word/syllable wherever you want (no
   matter whether the attachment automatism is on or off).

   Manipulations to the posted word/syllable have to be placed {\em
   before\/} the \ci{lyr} command.
 \item[Left justification.] Say
   \begin{quote}\ci{llyr}\end{quote}
   in order to have the next word/syllable left-justified to the
   notehead instead of centered.

   If you prefer to have left justification as the default, say
   \begin{quote}\ci{leftlyrtrue}\end{quote}
   before your piece begins (\ci{leftlyrfalse} to restore centered
   lyrics).
 \item[Word extension underlines.] The melisma commands (see
   \ref{melismas}) automatically manage word extension underlines;
   however, they can be set manually: Say
   \begin{quote}\ci{lyrrule}\end{quote}
   before the respective note command to initiate an underline {\em
   after\/} the word, and
   \begin{quote}\ci{lyrruleend}\end{quote}
   before the note where the underline shall be terminated.

   Note that no underline will be drawn if the word is continued
   afterwards so that the underline would interfere with the
   hyphenation.
 \item[Horizontal offset.]\label{lyroffset}%
   To shift a word/syllable out of its default
   horizontal position, say
   \begin{quote}\ci{lyroffset}\verb|{|\var n\verb|}|\end{quote}
   which --- similar to \verb|\roffset| --- moves the next word/syllable
   \var n quarter note head widths to the right (left if \var n
   is negative).

   Note that the effect of \ci{lyroffset} may be overridden by
   automatic lyrics shifting. If you want to gain complete manual
   control over the positioning, you have to disable the shifting
   automatism; see section \ref{horizpos} for details.
 \item[Extra lyrics.] If, as in the following
   example\footnote{Excerpt from ``In the town where I was born''
   ({\sl Eine kleine Melodie,\/} Strube, M\"unchen 1989)},

   \begin{music}
    \resetlyrics
    \instrumentnumber2
    \generalsignature1
    \setclef1\bass
    \songtop2
    \songbottom1
    \interinstrument=6mm
    \setsongraise2{-0.5mm}
    \scale{1.5}
    \forcelyrhyphenstrue
    \small
    \setlyrics{lower}{yel-low sub-ma-rine,} \assignlyrics1{lower}
    \copylyrics{lower}{upper}               \assignlyrics2{upper}
    \startextract
     \Notes\ibl0N0\zqbp0N\ibu1N0\qbp1b&%
	   \ibl2f0\zqbp2d\ibu3g0\qbp3g\en
     \notes\tbbl0\tbl0\zqb0N\tbbu1\tbu1\qb1b&%
	   \tbbl2\tbl2\zqb2d\tbbu3\tbu3\qb3g\en
     \Notes\ibl0M0\zqbp0M\ibu1N0\qbp1b&%
	   \ibl2f0\zqbp2d\ibu3g0\qbp3g\en
     \notes\tbbl0\tbl0\zqb0M\tbbu1\tbu1\qb1b&%
	   \tbbl2\tbl2\zqb2d\tbbu3\tbu3\qb3g\en
     \Notes\ibl0L{-1}\zqbp0L\hu b&\zhl d\hu g\en
     \notes\tbbl0\tbl0\lyric{ya,}\qb0K\en
     \Notes\ibl0L1\lyric{ya,}\qbp0L\en
     \notes\tbbl0\tbl0\lyric{ya,}\qb0M\en
    \endextract
   \end{music}
   \oneversespace

   a voice has ``extra'' lyrics (such as ``hmm'', ``ah'', ``ya'', etc.)
   in addition to the ``main'' lyrics, you need not enter them within
   \ci{setlyrics} (which would inhibit using \ci{copylyrics});
   instead, you can enter them directly within the music code using
   the command
   \begin{quote}\ci{lyric}\verb|{|\var{extra syllable}\verb|}|\end{quote}
   before the note where they shall appear. (Note that the effect is
   different from using \verb|\csong| which would not apply automatic
   hyphenation, horizontal positioning control, and \musixlyr's
   vertical positioning scheme.) Automatic lyrics attaching --- if
   active --- is suspended at that note. All manipulations can be
   applied; they must be given before the \ci{lyric} command.

   If you want the syllable linked to the following one by a hyphen,
   use \ci{lyrich} instead of \ci{lyric}.

   It may sometimes be necessary to post lyrics where no note is
   present at all, i.\,e.~in the empty space between notes. This is,
   however, possible with \verb|\lyric|, but is is unsuitable because
   automatic lyrics attachment shall {\em not\/} be suspended at the
   next note. In this case one of the commands \ci{lyric*} or --- for
   hyphenated syllables --- \ci{lyrich*} should rather be used; they
   have no effect on the ``regular'' flow of lyrics.
   \label{lyric*-verse-numbers}%
   One application for this are vertically aligned verse numbers. For
   example,

   \begin{quote}
     \bci{lyric*}
     \bci{verses}
     \begin{verbatim}
\setlyrics1{Spring time}
\setlyrics2{Au-tumn}
\assignlyrics1{1,2}
...
\znotes\loffset{2.2}{\verses{\lyric*{1.},\lyric*{2.}}}\en
\hardspace{4mm}%
\NOTes\hu{ee}\en
...\end{verbatim}
   \end{quote}

   yields:

   \begin{music}
     \resetlyrics
     \setlyrics1{Spring time}
     \setlyrics2{Au-tumn}
     \assignlyrics1{1,2}
     \lyrraise1{b -1.5mm}
     \scale{1.2}
     \startextract
       \znotes\loffset{2.2}{\verses{\lyric*{1.},\lyric*{2.}}}\en
       \hardspace{4mm}%
       \NOTes\hu{ee}\en
     \endextract
     \twoversesspace
   \end{music}

   (See section~\ref{multilyrlines} about attaching multiple verses
   to a single staff, a feature that has not been introduced so far.)

   You may even obtain verse numbers that are repeated at the
   beginning of each system by enclosing the \verb|\znotes...| command
   for setting the numbers in \verb|\def\everystaff{...}|.
   
   Another possible application --- combined with empty syllables ---
   is coping with unusual lyrics situations at voltas; for example,

   \begin{music}
     \twoversesspace
     \setlyrics1{first pass}
     \setlyrics2{se-cond pass}
     \assignlyrics1{1,2}
     \resetlyrics
     \scale{2}
     \startextract
       \NOTEs\wh g\en
	     \Setvolta1\def\atnextbar{\znotes\verses{,\lyric{}}\en}\bar
       \NOTEs\wh g\en
       \znotes\verses{,\lyrich*{}}\en \Setvolta2\setendvoltabox\rightrepeat
       \NOTEs\verses\lyricsoff\pause\en\bar
       \NOTEs\ha{gg}\en\setendvolta
     \endextract
     \twoversesspace
   \end{music}

   is accomplished by:

   \begin{quote}
     \bci{verses}
     \bci{lyric}
     \bci{lyrich*}
     \begin{verbatim}
\setlyrics1{first pass}
\setlyrics2{se-cond pass}
\assignlyrics1{1,2}
\startextract
  \NOTEs\wh g\en
  \Setvolta1\def\atnextbar{\znotes\verses{,\lyric{}}\en}\bar
  \NOTEs\wh g\en
  \znotes\verses{,\lyrich*{}}\en
  \Setvolta2\setendvoltabox\rightrepeat
  \NOTEs\verses\lyricsoff\pause\en\bar
  \NOTEs\ha{gg}\en\setendvolta
\endextract\end{verbatim}
   \end{quote}

   The trick used here is to inject empty syllables into the second
   verse at the barlines around the first volta; they serve as anchors
   for the hyphenation suspended under this volta.
   
   Note that the \ci{lyric}, \ci{lyric*}, \ci{lyrich}, and
   \ci{lyrich*} commands cannot be used at staves without assigned
   lyrics lines.
\end{description}



\subsection{Horizontal positioning}
\label{horizpos}

\musixlyr\ takes care of blank space between words. If a word --- at
its default position --- would come too close to or overlap the
previous word, it is shifted right in order to ensure a minimum of
blank space. This minimal interword space can be set by
\begin{quote}\ci{minlyrspace}\verb|=|\var{any \TeX\ dimension}\end{quote}
and is initially set to~3\,pt.

%The situation is different if two syllables belong to the same
%word. Such syllables are allowed to close up; only overlapping is
%inhibited by shifting. If the space between such syllables is
%narrower than a hyphen, the hyphen is omitted unless you force
%hyphens in every spatial situation saying
%\begin{quote}\verb|\forcelyrhyphenstrue|\end{quote}
%(to restore the default behaviour: \verb|\forcelyrhyphensfalse|). In
%this case, enough room is made for the hyphen by shifting the right
%syllable if necessary.

\leavevmode\label{nicerhyphens}%
The situation is different if two syllables belong to the same
word. Such syllables are allowed to close up; only overlapping is
inhibited by shifting. If the ``natural'' space between such syllables is
narrower than a hyphen, the hyphen is omitted and the two syllables
close up completely (eventually by left-shifting the second syllable
to prevent a small gap). You may though force
hyphens in every spatial situation saying
\begin{quote}\ci{forcelyrhyphenstrue}\end{quote}
(to restore the default behaviour: \ci{forcelyrhyphensfalse}). In
this case, enough room is made for the hyphen by right-shifting the 
second syllable if necessary.

If the note spacing is quite narrow, lyrics shifting may get
accumulated within a system. In order to prevent shifting words too
far out of their original position, a wider note spacing should be
considered. The command
\begin{quote}\ci{showlyrshifttrue}\end{quote}
marks each lyric shift with a black rectangle of the respective
width, thus making it easy to detect questionable positionings.

If you prefer to completely disable lyrics shifting, set
\ci{minlyrspace} to a large negative value (e.\,g.~$-10$\,cm).



\subsection{Vertical positioning}
\label{vertical-positioning}

By default, \musixlyr\ places lyrics vertically centered between the
two adjacent staves. Lyrics below the lowest or above the topmost
staff of a system are centered around the system's bottom or top
margin line.%
\footnote{This is not necessarily the most reasonable way, but it is
	  compatible to \musixtex's \printcmd{csong} behaviour.}

In most scores, you will have to deviate from this standard and
manually adjust the lyrics' vertical position. This can be done per
staff with the command
\begin{quote}
  \ci{lyrraise}\verb|{|\var n\verb|}|\verb|{|\var{pos}\var{raise}\verb|}|
\end{quote}
which applies the following setting to the lyrics attached to staff
number~\var n: \var{pos} is one of the letters \verb|a|
and~\verb|b|. \verb|a| sets the lyrics above the staff, \verb|b| below
the staff. \var{raise} is a dimension value by which the lyrics are
lifted above (or lowered under, if it is negative) the default
position as mentioned above. \ci{lyrraise}\verb|{|\var n\verb|}{b0pt}|
is the initial setting for all staves.

It is possible to apply changes to the vertical position at any place
within the score, but in most cases it is most reasonable to do so
within \verb|\atnextline{...}| which ensures consistent, even spacing
within each system.

For the sake of backward compatibility, applying \musixtex's command
\ci{setsongraise}\verb|{|\var n\verb|}{|\var{raise}\verb|}|
implicitly performs a
\ci{lyrraise}\verb|{|\var n\verb|}|\verb|{b|\var{raise}\verb|}|,
but not vice versa --- i.\,e.~\ci{lyrraise} has no effect on lyrics
set with \musixtex's lyrics commands \verb|\csong|,
\verb|\zsong|,~etc. Likewise, the obsolete \musixlyr\ command
\ci{auxsetsongraise} implies \verb|\auxlyr{\lyrraise...{b...}}|, but
not vice versa (see section~\ref{auxlyrics} to learn about the meaning
of \verb|\auxlyr|).



\subsection{Setting lyrics layout parameters}
\label{layoutpar}


\subsubsection{Available parameters}

\begin{description}
 \item[Font:] There is no special font command; the current font is
   used for lyrics. It can be changed using the usual \TeX/\LaTeX\
   commands.
 \item[Minimum length of word extension underlines:] A word extension
   underline is omitted if it would become too short. The minimum
   length can be set by
   \begin{quote}\ci{minlyrrulelength}\verb|=|\var{any \TeX\ dimension}
   \end{quote}
   and is initially set to~2\,mm. If you prefer to completely inhibit
   the underlines, set this parameter to an unsurpassably high value
   (e.\,g.~the width of your paper).
 \item[Minimum inter-syllable distance for multiple hyphens:] If the
   space between two hyphenated syllables (or between one of the
   syllables and the respective system margin) is wider than this
   distance, it is filled with a line of hyphens instead of a single
   hyphen. The minimum distance to be filled with more than one
   hyphen  can be set by
   \begin{quote}\ci{minmulthyphens}\verb|=|\var{any \TeX\ dimension}\end{quote}
   and is initially set to~15\,mm. If you prefer to completely inhibit
   filling with multiple hyphens, set this parameter to an
   unsurpassably high value.
 \item[Hyphen character:] This can be redefined by:
   \begin{quote}\verb|\def|\ci{lyrhyphenchar}\verb|{|\dots\verb|}|\end{quote}
\end{description}
Further layout parameters discussed elsewhere are: vertical
positioning (section \ref{vertical-positioning}), word justification
(section \ref{miscmanip}), minimum interword distance, and forcing
hyphens (section \ref{horizpos}).

{\bf Note:} In order to change the mentioned parameters globally, the
respective commands must be issued {\em outside\/}
\verb|\notes...\enotes|.

\subsubsection{Individual settings per staff}
\label{indivlayout}

All layout parameters can be set either globally (applying to all
lyrics lines) or individually, i.\,e.\ valid for a single specified staff.
An individual change is accomplished by enclosing the proper command
in
\begin{quote}\ci{lyrlayout}\verb|{|\dots\verb|}|\end{quote}
and issuing this within \verb|\notes...\enotes| in the section of the
referred-to staff. For example,
\begin{quote}
 \verb|\notes...&|\ci{lyrlayout}\verb|{\it\leftlyrtrue}...&...\en|
\end{quote}
applies italic font and left justification to the lyrics line(s)
assigned to the second staff.

Applying \ci{lyrlayout} to the same staff again overrides
{\em all\/} previous individual settings for this staff;
pay attention to this if you have made several individual settings and
want to reset some but retain others. Saying \ci{lyrlayout}\verb|{}|
restores the global settings for a staff.

The \ci{lyrlayout} command can also be used within
\ci{setlyrics}; this is useful if an individual setting
(especially a font change) should apply to a particular
portion of lyrics rather than to a portion of music. For example,
\begin{quote}
 \ci{setlyrics}\verb|{x}{An |\ci{lyrlayout}%
 \verb|{\it}em-pha-sized\lyrlayout{} phrase}|
\end{quote}
causes the word ``emphasized'' to appear italic wherever it is posted
within the music. (Be sure to attach \ci{lyrlayout} commands to
the corresponding lyrics portions {\em without\/} blank spaces in
between.)

{\bf As an exception,} the {\bf vertical positioning of lyrics} cannot be
changed within \ci{lyrlayout}; instead, it is handled staff-wise
using the special commands discussed in
section~\ref{vertical-positioning}.



\subsection{Multiple verses per staff}
\label{multilyrlines}

An arbitrary number of verses can be assigned to a staff
straightforwardly by saying
\begin{quote}\ci{assignlyrics}\verb|{|\var n\verb|}{|%
	     \var{names}\verb|}|\end{quote}
where \var{names} is a {\bf comma-separated list} of the desired
lyrics line names, listed from top to bottom. In the following
example\footnote{Excerpt from: ``Non moriar sed vivam'', motet by
{\sc Martin Luther}},

\begin{music}
 \resetlyrics
 \generalmeter\allabreve
 \generalsignature2
 \scale{0.9}
 \setlyrics{latin} {Non mo-ri-ar sed}
 \setlyrics{german}{Ich wer-de nicht ster-ben, son-dern}
 \assignlyrics1{latin,german}
 \small
 \setsongraise1{-1.5mm}
 \startextract
  \znotes\verses{,\lyrlayout{\footnotesize\it}}\en
  \NOTEs\wh{'a}\en\bar
  \NOTes\verses{\beginmel}\ha{'b}\verses{\endmel}\ha a\en\bar
  \NOtesp\beginmel\qup{'a}\en
  \Notes\ca{'b}\en
  \NOtes\verses{,\endmel}\qa{'c}\verses{,\beginmel}\itieu0d\qa d\en\bar
  \NOtes\ttie0\qa{'d}\endmel\qa c\en
  \NOtes\verses{\beginmel}\ha{'d}\en\bar
  \NOTes\verses{\endmel}\ha{'ee}\en
 \endextract
\end{music}
\twoversesspace

the lyrics lines have been set up as follows:
\begin{quote}
 \bci{assignlyrics}
 \begin{verbatim}
\setlyrics{latin} {Non mo-ri-ar sed}
\setlyrics{german}{Ich wer-de nicht ster-ben, son-dern}
\assignlyrics1{latin,german}\end{verbatim}
\end{quote}
 

\subsubsection{Handling verses individually}

If you code lyrics attachment manipulations the way
described in section~\ref{manipulation}, these manipulations apply
{\em to all lyrics lines\/} of the referred staff {\em
simultaneously.\/} In order to address each lyrics line separately,
use the command
\begin{quote}
 \ci{verses}\verb|{|\var{man1}\verb|,|\var{man2}\verb|,|\dots\verb|}|
\end{quote}
where \var{man1} is a sequence of commands to manipulate the upper
lyrics line, \var{man2} to manipulate the second lyrics line from
above, and so on. These command sequences may be empty. Delimiting
commas after the last non-empty list entry can be omitted. For example,
\begin{quote}
 \ci{verses}\verb|{,,\beginmel,\golyr{abc}}|
\end{quote}
leaves the upper two lyrics lines unaltered, initiates a melisma at
the third, performs a jump to label {\tt abc} at the fourth, and
leaves all further ones (whatever their number may be) unaltered. As
another example, the font change for the german lyrics of the above
excerpt is accomplished by
\begin{quote}
 \verb|\znotes|\ci{verses}\verb|{,\lyrlayout{\footnotesize\it}}\en|
\end{quote}
before the first note is coded.


\subsubsection{Vertical positioning}
\label{multiple-vert-pos}

A block of multiple lyrics lines is vertically centered around the position
which a single lyrics line would have taken; thus you can do things
like

{\small\setlyrstrut
 \resetlyrics
 \setlyrics{Winter1}{Win-ter a-de\rlap{,} a-ber dein Schei-den macht\rlap{,}}
 \setlyrics{Winter2}{schei-den tut weh\rlap,}
 \assignlyrics1{Winter1,Winter2}
 \generalmeter{\meterfrac34}
 \generalsignature2
 \setsongraise1{-2mm}
 \startextract
  \NOtes\qa{ffe}\en\bar
  \NOTesp\hup d\en\rightrepeat
  \assignlyrics1{Winter1}%
  \NOtes\qa{fgh}\en\bar
  \NOtes\qa h\en
  \Notes\beginmel\Ibu0gf1\qb0g\endmel\tqh0f\en
  \NOtes\qa g\en
 \endextract}
\twoversesspace

without having to adjust the vertical position within the
piece. (See the source of this manual for the coding of this
example.)

\leavevmode\label{lyrstrut}%
The vertical distance between the baselines of multiple verses is
appropriate for the default 10~points font size of plain \TeX.
However, the size-switching commands \verb|\smalltype|,
\verb|\medtype|~etc.\ do not implicitly adjust the vertical distance
to the chosen font size. Since the vertical distance of lyrics is
controlled by the concept of a {\em lyrics strut\/}\footnote{In
analogy to \TeX's {\tt\char92 strut}; see {\sl The \TeX book\/} by
{\sc D.~E.~Knuth.}}, you should perform this adjustment explicitly with
the command
\begin{quote}\ci{setlyrstrut}\end{quote}
directly after each ``global'' font size change (i.\,e.~after each size
change not issued within \verb|\lyrlayout{...}|;
see~\ref{indivlayout}) unless you do not want the spacing to be
adjusted. (\TeX perts only: If you like to define your own vertical
spacing, you may redefine the \ci{lyrstrutbox} yourself.)



\subsection{Auxiliary lyrics}
\label{auxlyrics}

If you have typeset two voices of a polyphonic arrangement within one
staff, you normally want to give each voice its own lyrics lines.
Usually, the two voices are distinguished by means of the notes' stem
direction; the lyrics of the up-stemmed voice are typeset above,
those of the down-stemmed voice below the staff. \musixlyr\
provides the facility to maintain such two lyrics ``streams'' per
staff; one of them --- which is handled exactly as described so
far --- will be called {\em main lyrics,\/} the second one {\em
auxiliary lyrics.\/}


\subsubsection{Addressing auxiliary lyrics}

Only two rules are to be obeyed for directing commands to auxiliary lyrics:
\begin{enumerate}
 \item Enclose everything concerning auxiliary lyrics in:
   \begin{quote}\ci{auxlyr}\verb|{...}|\end{quote}
   For example, assign lyrics line~{\tt verse1} to staff~2 as
   auxiliary lyrics by:
   \begin{quote}\ci{auxlyr}\verb|{\assignlyrics{2}{verse1}}|\end{quote}
   For example, begin a melisma at the next note of the ``auxiliary'' voice by:
   \begin{quote}\ci{auxlyr}\verb|{\beginmel}|\end{quote}
   (For \TeX perts: If the argument of \ci{auxlyr} consists of
   only one token, the braces can be omitted, so that
   \verb|\auxlyr\beginmel| does the same.)
 \item There are exactly two exceptions to the first rule. The first
   one are the \ci{setlyrics}, \ci{appendlyrics}, and \ci{copylyrics}
   commands which need not be enclosed in \ci{auxlyr}\verb|{...}|;
   their function is independent of the later use of the defined
   lyrics lines as either main or auxiliary lyrics.

   The second exception concerns lyrics commands issued as type-1
   \TeX\ strings in a \PMX\ source; see below for details.
\end{enumerate}

\label{auxlyr-vertical-position-default}%
Auxiliary lyrics are lifted above the staff by default, i.\,e.~they have
the same vertical position as main lyrics belonging to the staff above
the current staff. If this causes collisions with the main lyrics of
the upper staff, you have to shift the lyrics away from each other
using the vertical spacing commands discussed in
section~\ref{vertical-positioning}.


\subsubsection{Automatic attachment of auxiliary lyrics}
\label{auxlyr-attachment}

While it is easy to attach only one lyrics ``stream'' to a single
voice within a system automatically, the attachment of both main and
auxiliary lyrics involves the task of determining which note belongs
to which voice. Unfortunately, \musixtex\ does not explicitly state a
note's membership of a particular voice; therefore, a heuristic
method has to be used to gather this information. However,
the situation is completely different if you prepare your scores with
the \PMX%
\footnote{If you don't know what \PMX\ is, have a look at:
	  {\tt http://icking-music-archive.sunsite.dk/musixtex/software/pmx/}}
preprocessor. The following paragraphs explain managing the lyrics
attachment automatism for both cases --- \PMX\ and
non-\PMX\ --- separately.


\paragraph{Using \PMX.}
\label{pmx-autodetection}

(This paragraph requires some \PMX\ knowledge.) For \PMX-prepared
scores (which \musixlyr\ detects automagically\footnote{%
  This automatic \PMX\ detection may cause problems for processing
  \PMX\ scores prepared with the \PMX-unaware \musixlyr\
  version~1.0. See section~\ref{no-pmx-autodetection} about how to
  solve this backward-incompatibility issue.
}), the distinction
between the two voices of a staff works fully automatic. This applies
to both the automatic lyrics attachment as well as explicit lyrics
manipulations --- i.\,e.,~you do not have to use \ci{auxlyr}\verb|{...}| for
manipulating auxiliary lyrics. (This only applies to manipulations
that appear in a voice's context, i.\,e.~within a type-1 \TeX\ string.)
For reference, see the following example\footnote{Excerpt from
``Waldesnacht, du wunderk\"uhle'' by {\sc Johannes Brahms} ({\sl Eine
kleine Melodie,\/} Strube, M\"unchen 1989)},

\oneversespace
\begin{music}
 \resetlyrics\small
 \setlyrics{soprano}{all der ir-ren Qua-len los.}
 \auxlyr{\assignlyrics1{soprano}}
 \copylyrics{soprano}{alto}
 \assignlyrics1{alto}

 \auxlyr{\lyrraise1{a -3\internote}}
 \staffbotmarg5.5\Interligne
 \stafftopmarg5\Interligne
 \generalsignature2
 \lyrmodealter0
 
 \startextract\addspace\afterruleskip
  \scale{1.2}%
  \NOtes\zhu{=j}\lower6\internote\qp\lifthpause{-3}\sk\qu h\en
  \bar
  \NOtes\auxlyr\beginmel\roffset{.8}{\isluru2j}\zhup f\lower8\internote\qp\en
  \NOTes\hl{^d}\en
  \NOTes\zql e\itieu0e\qu e\en
  \bar
  \NOtes\beginmel\islurd1c\zql c\ttie0\tslur2h\auxlyr\endmel\qu e%
	\roff{\endmel\tsslur1d\zhl d}\qu{=d}%
	\auxlyr\beginmel\itieu0d\hu d%
	\ql d\en
  \bar
  \NOTes\zhl d\ttie0\auxlyr\endmel\hu d%
	\zhl a\hu c\en
  \bar
  \NOTes\zhl a\fermataup l\hu d\hpause\en
 \setdoubleBAR\endextract
\end{music}
\oneversespace

which can be obtained by the following \PMX\ code (processed with
\PMX\ version~2.3):
\begin{quote}
\bci{auxlyr}
\begin{verbatim}
---
\input musixtex                     % These commands are necessary
\input pmx                          % because pmx.tex MUST be \input
\input musixlyr                     % before musixlyr.tex!

\setlyrics{soprano}                 % \setlyrics and \copylyrics
  {all der ir-ren Qua-len los.}     %  need not be enclosed in
\copylyrics{soprano}{alto}          % \auxlyr{...}.

\assignlyrics1{alto}
\auxlyr{\assignlyrics1{soprano}
        \lyrraise1{a 3\Interligne}}

\let\BM\beginmel                    % For brevity within the subsequent
\let\EM\endmel                      % music coding recommended.
---
1 1 4 4 0 0 3 2 0 5 20 0

t
./

% 1st and 2nd bar: 
r4-6 r2-6    |       r4-8    d24s             e4 //
c25n      a4 | \BM\ f2d s+4       \isu0e{.8}\ e4 /
% 3rd bar:
\BM\        c s   \EM\ d2r s       d4   //
\ts0e0\EM\  e s+3      dn     \BM\ d2 s /
% 4th bar:
      d2  a //                            % Note that \BM and \EM
\EM\  d s c /                             % apply either to main
% 5th bar:                                % or to auxiliary lyrics,
a    r+0 //                               % depending on which voice
d of rb  /                                % they appear in.\end{verbatim}
\end{quote}

By default, main lyrics are attached to the lower voice (whose
lines end with \verb|//| in the \PMX\ source), auxiliary lyrics to the
upper voice; due to \PMX\ internals, this ensures that if only one
voice is present, main lyrics are attached to it. However, situations
exist in which it is desirable to have main lyrics attached to the
upper voice, e.\,g.~when you have only main lyrics and want to
synchronize them with the soprano voice instead of the alto voice
(both of them set on the same staff). In this case, you can toggle the
attachment of lyrics to voices with the command
\begin{quote}\ci{lyrmodealter}\verb|{|\var n\verb|}|\end{quote} for
staff number \var n; the command
\begin{quote}\ci{lyrmodenormal}\verb|{|\var n\verb|}|\end{quote}
restores the default behaviour. If $\hbox{\var n}=0$ then the change
affects all staves.


\paragraph{Using ``plain'' \musixtex\ without \PMX.}

In this context, there are two basically different modes for lyrics
attachment:
\begin{enumerate}
 \item In {\em normal mode\/} (which is active by default) only main
   lyrics are provided automatically as described in
   section~\ref{auto-attachment}. All auxiliary lyrics must be issued
   manually by saying \ci{auxlyr}\ci{lyr} before the respective note.
   This mode is useful when you do not have auxiliary lyrics at all
   or when auxiliary lyrics occur only occasionally within the piece.
 \item In {\em alternative mode\/}, both main and auxiliary lyrics
   are provided automatically. The two voices of a staff are
   distinguished by means of their notes' stem direction; the lower
   voice gets main lyrics, the upper one auxiliary lyrics. This
   implies that the automatism fails at stemless
   notes\footnote{Ancient notes, namely breve, longa, and maxima, are
   also treated as stemless notes by \musixtex's internals.}, thus you
   have to provide lyrics manually by saying \ci{lyr} or
   \ci{auxlyr}\ci{lyr}, respectively, before such notes. This mode is
   meant for staves which continuously contain two polyphonic voices
   with separate lyrics.
\end{enumerate}

The appropriate mode can be chosen staff-wise: The command
\begin{quote}\ci{lyrmodealter}\verb|{|\var n\verb|}|\end{quote}
enables alternative mode for staff number \var n; the command
\begin{quote}\ci{lyrmodenormal}\verb|{|\var n\verb|}|\end{quote}
restores normal mode. As stated above, $\hbox{\var n}=0$ performs the
choice affecting all staves.

For illustration, have a look at the following
example\footnote{Excerpt from ``Audite nova'' by {\sc Orlando di
Lasso} ({\sl ars musica\/} vol.\,4, M\"oseler Verlag
Wolfenb\"uttel/Z\"urich)}:

\oneversespace
\begin{music}
 \resetlyrics \small
 \lyrmodealter0
 \setlyrics{soprano}{bring her die Gans,} \auxlyr{\assignlyrics1{soprano}}
 \copylyrics{soprano}{alto}                       \assignlyrics1{alto}
 
 \generalsignature{-2}
 \advance\stafftopmarg3\Interligne
 \advance\staffbotmarg2\Interligne
 
 \startextract\addspace\afterruleskip
  \NOtes\zqu g\ql e\en
  \bar
  \Notes\zqu i\beginmel\ibslurd0f\ibl0f{-1}\qb0f\en
  \Notes\qb0e\en
  \Notes\auxlyr\beginmel\ibsluru1i\zqup i\qb0d\en
  \Notes\tqb0e\en
  \Notes\endmel\tbslurd0e\ql c\en
  \notes\ibbu0h{-1}\qb0h\en
  \notes\tqh0g\en
  \NOtes\auxlyr\endmel\tbsluru1h\zqu h\ql c\en
  \bar
  \NOTEs\auxlyr\lyr\zwh i%       The lyrics of the whole notes
	\lyr\wh b\en     %       must be given manually.
 \endextract
 \lyrmodenormal0
\end{music}
\oneversespace

The lyrics were defined as follows:
\begin{quote}
 \bci{auxlyr}
 \bci{assignlyrics}
 \begin{verbatim}
\setlyrics{soprano}{bring her die Gans,}
\copylyrics{soprano}{alto}

\assignlyrics1{alto}
\auxlyr{\assignlyrics1{soprano}}

\lyrmodealter0                    % provide main + aux. lyrics automatically\end{verbatim}
\end{quote}

This is the complete notes section:
\begin{quote}
 \bci{auxlyr}
 \begin{verbatim}
\NOtes\zqu g\ql e\en
\bar
\Notes\zqu i\beginmel\ibslurd0f\ibl0f{-1}\qb0f\en
\Notes\qb0e\en
\Notes\auxlyr\beginmel\ibsluru1i\zqup i\qb0d\en
\Notes\tqb0e\en
\Notes\endmel\tbslurd0e\ql c\en
\notes\ibbu0h{-1}\qb0h\en
\notes\tqh0g\en
\NOtes\auxlyr\endmel\tbsluru1h\zqu h\ql c\en
\bar
\NOTEs\auxlyr\lyr\zwh i%       The lyrics of the whole notes
      \lyr\wh b\en     %       must be given manually.\end{verbatim}
\end{quote}

Note that, as opposed to using \PMX, there is no special provision
for attaching main lyrics to the upper voice. If you desire a single
lyrics ``thread'' synchronized with the upper voice, please use
auxiliary lyrics instead and position them below the staff (see
section~\ref{vertical-positioning}).
Alternatively, you may use main lyrics in normal lyrics mode and code
the lower voice completely non-spacing (in the sense explained
in~\ref{auto-attachment}), the upper voice completely spacing; this
should not be too complicated for mainly homophonic pieces.



\subsection{Lyrics at multi-staff instruments}
\label{multi-staff-instruments}

So far in this manual, the term {\em staff\/} has always been used for
what \musixtex\ precisely calls an {\em instrument\/}. This section,
however, correctly adopts the \musixtex\ terminology.

Lyrics can be assigned to staves of a multi-staff instrument. To do
so, use
\begin{quote}
  \ci{assignlyricsmulti}\verb|{|\var i\verb|}{|\var s\verb|}{...}|
\end{quote}
instead of \verb|\assignlyrics...|; here, \var s~addresses the desired
staff number (counted from bottom to top) of instrument~\var i.

Nevertheless,
\ci{assignlyrics}\verb|{|\var i\verb|}{...}|
can be used as well if instrument~\var i has multiple staves; in this
case, it is equivalent to
\verb|\assignlyricsmulti{|\var i\verb|}{1}{...}|,
i.\,e.~it affects only the first staff (and not all staves, as may be
presumed).

In the same spirit, the following ``multi'' commands replace the
standard ones within a multi-staff instrument context:
\begin{quote}
  \ci{lyrraisemulti}\\
  \ci{lyrmodenormalmulti}\\
  \ci{lyrmodealtermulti}
\end{quote}

All other \musixlyr\ commands or features can be used the same way as
within a single-staff instrument context.

Using lyrics at multi-staff instruments can be useful, for example,
for typesetting songs with piano accompaniment:%
\footnote{Example taken from songbook {\sl Am Brunnen vor dem Tore\/},
Delphin Verlag M\"unchen/Z\"urich 1981}

\begin{music}
 \resetlyrics
 \setlyrics{verse}{Gu-ter Mond, du gehst_ so_ stil_-le}
 \assignlyricsmulti11{}
 \assignlyricsmulti12{verse}
 \lyrraisemulti12{b 2\internote}
 \setstaffs12
 \generalsignature{-1}
 \generalmeter\meterC
 \setclef1\bass
 \interstaff{13}
 \scale{1.3}
 \startextract
  \Notes\qp|\ibl0j{-2}\qb0j\tqb0i\en
  \bar\hardspace{1.7\myqnwidth}%
  \NOtes\isluru1d\lq c\zq M\ql d|\qa h\en
  \NOtes\zq c\zq M\ql e|\qa h\en
  \Notes\lq c\zq M\ql d|\islurd0h\ibu0h2\qb0h\tslur0i\tqh0i\en
  \Notes\tslur1e\zq c\zq M\ql e|\isluru0j\ibl0j2\qb0j\tslur0k\tqb0k\en
  \bar\hardspace\myqnwidth
  \Notes\isluru1d\lq c\zq M\ql d|\isluru0j\ibl0j{-2}\qb0j\tslur0i\tqb0i\en
  \NOtes\zq c\zq M\ql e|\qa g\en
  \NOtes\lq c\zq M\ql d|\qp\en
  \znotes\midslur2\tsslur1d\en
 \zendextract
\end{music}

The lyrics of this example are defined as follows:
\begin{quote}
 \bci{assignlyricsmulti}
 \bci{lyrraisemulti}
 \begin{verbatim}
\setlyrics{verse}{Gu-ter Mond, du gehst_ so_ stil_-le}
\assignlyricsmulti{1}{2}{verse}
\lyrraisemulti    {1}{2}{b \internote}\end{verbatim}
\end{quote}



\subsection{Managing lyrics assignment according to current context}
\label{assignment-by-context}

Managing the properties of lyrics lines assignment to instruments by
absolute instrument numbers --- as it is done with the
\ci{assignlyrics}, \ci{lyrraise}, and \ci{lyrmodealter}
commands --- may cause difficulties whenever the score's number and
order of instruments shall be rearranged; in this case instrument
numbers change, thus the instrument number parameters of the
mentioned lyrics-related commands have to be adjusted.

This is especially unpleasant where such a rearrangement is done
automatically, e.\,g.~by \PMX's part extraction facility {\em
scor2prt\/}; this is intended to work without any manual editing
of the generated part files.

To address this issue, \musixlyr\ provides {\em context-dependent\/}
variants of the commands mentioned above. For example: Instead of
\verb|\assignlyrics{2}{soprano}|, you may say something like:
\begin{quote}
  \verb|\notes...&|\ci{assignlyricshere}\verb|{soprano}...\en|
\end{quote}
\ci{assignlyricshere}\verb|{|\var{name(s)}\verb|}| assigns the given verse
\var{name(s)} to the instrument/staff whose context it occurs in within the
\verb|\notes...\en| construct; no absolute instrument/staff number needs to
be specified.

For example, the following two code fragments are equivalent:

\begin{quote}
 \footnotesize
 \hfill
 \def\arraystretch{1}%
 \begin{tabular}{l@{\qquad}|@{\qquad}l}
  \begin{tabular}[t]{@{}l@{}}
   \ci{assignlyrics}\verb|     {1}   {acct}%|\\
   \ci{assignlyricsmulti}\verb|{2}{2}{verse}%|\\
   \verb.\NOtes\qa c&\qa e|\qa g\en.
  \end{tabular}
  &
  \begin{tabular}[t]{@{}l@{}}
   \verb|\NOtes|\ci{assignlyricshere}\verb|{acct}\qa c%|\\
   \verb|     &\qa e%|\\
   \verb.     |.\ci{assignlyricshere}\verb|{verse}\qa g\en|
  \end{tabular}
 \end{tabular}
 \hfill\strut
\end{quote}

Using \PMX, the context-dependent commands must be given as type-1
inline \TeX\ strings. Here is the same example, coded as \PMX input:

\begin{quote}
 \footnotesize
 \hfill
 \def\arraystretch{1}%
 \begin{tabular}{l@{\qquad}|@{\qquad}l}
  \begin{tabular}[t]{@{}l@{}}
   \ci{assignlyrics}\verb|     {1}   {acct}\|\\
   \ci{assignlyricsmulti}\verb|{2}{2}{verse}\|\\
   \verb|c44 /|\\
   \verb|e44 /|\\
   \verb|g44 /|\\
  \end{tabular}
  &
  \begin{tabular}[t]{@{}l@{}}
   \hphantom{\tt .NOtes.assignlyricshere.acct..qa cg.en}\\
   \\
   \ci{assignlyricshere}\verb|{acct}\  c44 /|\\
   \verb|                          e44 /|\\
   \ci{assignlyricshere}\verb|{verse}\ g44 /|\\
  \end{tabular}
 \end{tabular}
 \hfill\strut
\end{quote}

The first one (left) results in wrong lyrics assignments in {\em
scor2prt\/} output; the second one (right) is robust against part
extraction.

This is the complete set of ``here'' commands that can be applied the
same way:
\begin{quote}
  \ci{assignlyricshere}\\
  \ci{lyrraisehere}\\
  \ci{lyrmodenormalhere}\\
  \ci{lyrmodealterhere}
\end{quote}

All other \musixlyr\ commands are context-driven anyway, i.\,e.~using
the mentioned ``here'' commands you can completely omit any absolute
instrument/staff number assignment.



\subsection{Miscellaneous}

\subsubsection{Avoiding problems after {\tt\char92
	       endpiece...\char92 startpiece}}

Sometimes, after having suspended the music with \verb|\endpiece| and
restarted it with \verb|\startpiece| (or the respective
\verb|\...extract| commands), the lyrics lines are totally misaligned
because the internal lyrics positioning parameters are not reset
properly. Saying
\begin{quote}\ci{resetlyrics}\end{quote}
before the beginning of the new music section should solve this
problem.


\subsubsection{Improved centering of words with punctuation marks}

Centered words with an attached punctuation mark often seem a bit
left-shifted since the punctuation mark also counts for the
centering. To have a word centered solely with respect to its
letters, enclose the punctuation mark in \verb|\rlap{...}| as it is
done in the example of section~\ref{multiple-vert-pos}. Unfortunately,
this circumvents the lyrics shifting automatism, and is incompatible
with the use of \ci{lyrnop}.


\subsubsection{Verse numbers}
\label{versenumbers}

There is no special command for this topic. See the following example:

\begin{music}
 \resetlyrics
 \generalmeter{\meterfrac24}
 \scale{1.8}
 \setlyrics{sailor1}
	   {\llap{1.} What shall we do with a drun-ken sail-or\rlap{?}}
 \setlyrics{sailor2}
	   {\llap{2.} Take him and shake him and try to\lyrlink a-wake him,}
 \setlyrics{sailor3}{\llap{3.} Give him a dose of_ salt and wa-ter,}
 \setlyrics{sailor4}{\llap{4.} Give him a dash with a bo-som's rub-ber,}
 \assignlyrics1{sailor1,sailor2,sailor3,sailor4}
 \small\setlyrstrut
 \setsongraise1{-6mm}
 \startextract
  \znotes\llyr\lyroffset{-2}%    make the numbers stand off to the left
	 \lyr%                   post the numbers
  \en
  \Notes\ibu0h0\qb0h\en
  \notes\nbbu0\qb0h\tqh0h\en
  \Notes\ibu0h0\qb0h\en
  \notes\nbbu0\qb0h\tqh0h\en\bar
  \Notes\Dqbu hd\Dqbu fh\en
 \endextract
\end{music}
\oneversespace
\twoversesspace

To see how this is accomplished, look for the example's coding within
{\tt mxlyrdoc.tex}.

Section~\ref{lyric*-verse-numbers} explains an alternative method for
setting vertically aligned verse numbers.


\subsubsection{``Lyrics slurs''}

If two consecutive syllables of a lyrics line share the same note (as
in the second verse of the above example), you may want to link them
together with a small slur under them \mbox{like\lyrlink this.} To do
this, use the command \ci{lyrlink} instead of the respective blank
space within the \verb|\setlyrics| definition.

If any of the adjacent characters extend under their \TeX\ baseline
(like `g', `j', `p', etc.), use \ci{lowlyrlink} which lowers the
slur accordingly (e.\,g.: \mbox{sing\lowlyrlink gladly}).


\subsubsection{Restoring the \PMX-unaware behaviour of \musixlyr~1.0}
\label{no-pmx-autodetection}

As stated in section~\ref{pmx-autodetection}, \musixlyr\ distinguishes
between main and auxiliary lyrics context in a special way when it
detects that the score has been set up using \PMX. However, this
automatic \PMX\ detection may cause problems for processing \PMX\
scores prepared with the \PMX-unaware \musixlyr\ version~1.0. To solve
this, you can force the old \musixlyr\ behaviour by an explicit macro
redefinition as follows:

Within {\tt musixlyr.tex}, find the code fragment
\begin{quote}\begin{verbatim}
\ifx\nextvoice\undefined
  \def\decide@lyrmode{%
    \call@lyrmode
    ...
    \fi\fi\fi\fi\fi\fi\fi}
\else\end{verbatim}\end{quote}
(which is about 25 lines long). Copy it, exluding the first and the
last line, to your score source and surround it by the
\verb|\makeatletter...\makeatother| command pair. This
should then look like this:
\begin{quote}\begin{verbatim}
\makeatletter
  \def\decide@lyrmode{%
    \call@lyrmode
    ...
    \fi\fi\fi\fi\fi\fi\fi}
\makeatother\end{verbatim}\end{quote}

To restore the \PMX-aware \musixlyr\ behaviour of version~1.1 and
newer, copy the {\em second\/} \verb|\decide@lyrmode| definition
from {\tt musixlyr.tex} to the desired place in your score and
surround it with \verb|\makeatletter...\makeatother|.

To do these redefinitions within a \PMX\ input file, it is probably
always best to wrap them in macro definitions within the \PMX\ file
preamble. This way, you can switch between the ``old'' and ``new''
\musixlyr\ behaviour by simply calling these macros in \PMX's
inline-\TeX\ style.

\pagebreak[4]
\section{Command index}
\printindex



\section{Software license}

\musixlyr\ (including this manual) is free software; you can
redistribute it and/or modify it under the terms of the GNU General
Public License as published by the Free Software Foundation; either
version~2 of the License, or any later version.

\musixlyr\ is distributed in the hope that it will be useful,
but WITHOUT ANY WARRANTY; without even the implied warranty of
MERCHANTABILITY or FITNESS FOR A PARTICULAR PURPOSE.  See the
GNU General Public License for more details.

You should have received a copy of the GNU General Public License
along with \musixlyr; if not, write to the Free Software
Foundation, Inc., 675 Mass Ave, Cambridge, MA 02139, USA.

\medskip
Author:
\begin{tabular}[t]{l}
 Rainer Dunker\\
 Wachtelweg 31\\
 85\,591 Vaterstetten\\[0.5ex]
 Germany\\[1ex]
 E-mail: \tt rainer.dunker@web.de
\end{tabular}

\end{document}
